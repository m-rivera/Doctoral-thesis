\chapter*{Conclusions}
\addcontentsline{toc}{chapter}{Conclusion}
\section*{Summary}
This thesis offered a detailed account of the development, implementation, and application of a new suite of tools for the study of excited states in molecular crystals. Initially, we were interested in applying the self-consistent Ewald embedding scheme of Wilbraham \textit{et. al}\cite{Wilbraham2016a} to our own set of proton transfer crystals, 2'hydroxychalcones, with a focus on the SLE mechanism. It quickly surfaced that due to the comparative flexibility of our compounds, a simple point charge embedding scheme would need to be complemented by short range interactions, in order to allow for a realistic exploration of the PES.

This led us to the development of the ONIOM Ewald Embedded Cluster models described in Chapter \ref{chap:jctc}. Therein, we used a hierarchy of models to recover the correct emissive behaviour of two crystals, HC1 and HC2, only the former of which displayed Solid State Luminescent Enhancement. Our new and most accurate method combined ONIOM QM:QM' with Ewald point charge embedding, in order to eliminate the electrostatic truncation error that accompanies traditional ONIOM QM:QM' in periodic systems. Not only were we able to correctly describe non-Born-Oppenheimer regions of the PES in order to describe Solid State Luminescent Enhancement in these materials using the Restricted Access to Conical Intersections model, we also found emission energies within 0.2 eV of the experimentally observed ones, a result overestimated both by QM:MM and traditional QM:QM'.

The implementation of a hybrid level of theory method into a program immediately hinted to the development of a programming library, since allowing the user a great degree of flexibility was paramount. This paved the way for the development of \texttt{fromage}, described in Chapter \ref{chap:prog}. Along with these ONIOM methods, many other tools were developed by our group since 2016 to aid in the study of molecular crystal excited states, and they were all implemented in \texttt{fromage}. Working with molecules, finite cluster models, and unit cells all at once started off as a very time intensive task from a practical point of view, because of the niche geometry manipulation operations which were ubiquitous as part of the workflow, and the decentralisation of post-processing analysis tools. \texttt{fromage} has streamlined these processes significantly by translating these geometrical objects into Python. This allowed for the development of tools to analyse characteristic dimer geometries within aggregates and manipulate different classes of clusters. It was also used to investigate excited state electronic structures, allowing for the classification of excitonic states, and the evaluation their exciton coupling.

With this arsenal of tools at hand, we broadened our scope, to investigate larger families of compounds. Chapter \ref{chap:molecules} offered a study of thirteen emissive molecular crystals, with different luminescent properties in solution. The competition between radiative and nonradiative decay pathways was explored, highlighting how both vibrational decay and internal conversion \textit{via} conical intersection needed to be quenched in the crystal, without introducing additional excitonic dissipation mechanisms. In our systems, low frequency normal modes were significantly quenched by crystallisation, disfavouring vibrational decay, regardless of its importance in vacuum. The conical intersection was systematically found to be level with or higher in energy than the absorption energy in crystal, thus blocking this channel by the means of steric hindrance. Conical intersections tended to involve puckering in the crystal, which was found to destabilised them more than rotational conical intersections. Energy dissipation through excitonic delocalisation was controlled for by comparing both exciton coupling values and reorganisation energies between different dimers in the crystal.

Of the objectives outlined in the introduction, the first three (extending system-environment interactions, identifying an optimal point charge scheme, probing the response of the environment) are addressed by Chapter \ref{chap:jctc}. The fourth objective, of producing a tailored implementation of these methods to molecular crystals is addressed by Chapter \ref{chap:prog}. The role of Chapter \ref{chap:molecules} is to demonstrate the use and robustness of the outcomes of the previous two chapters.

This doctoral project has therefore produced all of the key steps to enable the modelling of photochemical processes in molecular crystals within a level of accuracy which would previously have been impossible for certain systems. The \texttt{fromage} package is already in use on at least three continents, and will unlock new areas of study for future researchers. Readers can find the source code and documentation here:\\
\href{https://github.com/Crespo-Otero-group/fromage}{https://github.com/Crespo-Otero-group/fromage}\\
\href{https://fromage.readthedocs.io}{https://fromage.readthedocs.io}

\section*{Outlook}
\subsection*{Point Charge Description}
The point charge approximation for the potential emanating from an atom, upon which hinges the ONIOM QM:QM' formalisms described herein, has shown to reach its limits when combined with too diffuse densities. This limits the amount of systems for which these methods could be applicable, as well as the size of the basis set.

We can fathom a whole family of substitutes for the point charges of the molecules, each with its own advantages. A first solution would be to introduce a local repulsive term to the Coulomb interaction of each point charge, accounting for the Pauli repulsion of the atom. This would require parameterisation, but keep the complexity of the embedding scheme to a minimum. Alternatively, smoothing out the point charge potential as Gaussian functions, with a finite cusp, would limit unphysical solutions due to infinite potential wells, but would not account for atomic repulsion.

Embedding the excited state Hamiltonian higher orders of the multipole expansion would further help recover mutual polarisation more naturally than in the self-consistent scheme, whilst keeping the modelling cost to a minimum. Here, the implementation aspect becomes arduous due to the scarcity of electronic structure codes accepting this sort of embedding. Additionally, determining the value of the multipoles representing the environment is not trivial. This polarisation could also be achieved by combining the point charges with a polarisable continuum model,\cite{Loco2016} which would involve parameterising the dielectric constant of the material.

Finally, the interaction between system and environment could be treated in a Rayleigh-Shr\"odinger perturbative way, accounting for accurate Coulomb interactions, polarisation, dispersion, and exchange provided a symmetry-adapted formulation.\cite{Jeziorski1994,Jansen2014} The implementation of a Symmetry-Adapted Perturbation Theory to excited states with ONIOM has yet to be fully developed.

\subsection*{Geometry Optimisation}
Currently, the implementation of ONIOM schemes only accounts for the optimisation of region \textbf{1} of the crystal. Optimising region \textbf{2} would allow for the modelling of many-body reorganisation within the crystal, which could be critical in some systems.

Allowing for the optimisation of region \textbf{2} would require a careful consideration of the conservation of energy in the ONIOM equation, and importantly an Ewald embedding for this region.

Additionally, the current method for optimising conical intersections is the penalty function method. This has the advantage of forgoing the need for derivative and nonadiabatic coupling vectors. However those quantities are available in several packages already interfaced with \texttt{fromage}, meaning that they could be exploited to locate conical intersections faster and more precisely.

\subsection*{Ewald Potential}
Currently, the Madelung sum of the crystal is accounted for by varying point charge values to reproduce an Ewald potential. This is all handled by the program \texttt{Ewald},\cite{Klintenberg2000,Derenzo2000} which was initially devised for ionic systems with a quantum region much smaller than the clusters which are relevant to molecular crystals.

A new implementation of this Ewald scheme would benefit the program hugely, by being tailored to these cluster geometries, and removing implementation problems which arise at large length scales. Other schemes also exist to mimic the result of an Ewald sum which may require fewer point charges or computational power to match,\cite{Fukuda2012} such as the Wolf sum.\cite{Wolf1999}